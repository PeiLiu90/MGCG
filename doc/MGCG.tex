\documentclass{article}

\usepackage{indentfirst}
\setlength{\textwidth}{15cm}
\setlength{\textheight}{22cm}
\setlength{\topmargin}{-1cm}
\setlength{\oddsidemargin 1cm}
\setlength{\evensidemargin 1cm}
\setlength{\topskip 0cm}
\setlength{\footskip 1cm}

\usepackage{graphicx}

\usepackage{amssymb}
\usepackage{amsmath}
\usepackage{float}
\usepackage{bm}

\graphicspath{{../data/}}

\begin{document}

\title{Notes on Multigrid Preconditioned Conjugate Gradient Method}
%\author{Pei Liu\thanks{Department of Mathematics \& Institute of Nature Sciences, Shanghai Jiao Tong University}}
\maketitle

\section{Introduction}
For high dimensional(2D or 3D) Variable Coefficients Elliptic PDEs, we need to solve a large sparse linear equation. FFT is no longer valid due to the coefficients is dependent on coordinates.

Conjugate Gradient method is good for such linear system, especially with a proper preconditioner. People have proved multigrid method could be a good choice. The resulting algorithm is then called Multigrid Preconditioned Conjugate Gradient Method (MGCG).

\section{MGCG}
We have to notice that the preconditioner for CG have some constraints: positive definite and symmetric. So, not all multigrid method could be used. 

\subsection{Smoother}

the smoother need to be symmetric, which means Gauss-Seidel and SOR is excluded. We could use damped Jacobi, Red-Black Symmetric Gauss Seidel or Multi-Color Symmetric Successive Over-Relaxation.

I choose RBSGS for it is easy to parallelism. The key ideal is to decompose the grid into two set: one is label as 'red', the other is label as 'black'. 

For 2D elliptic equation:
\begin{equation}
-\nabla \epsilon \nabla u + k^2 u= f
\end{equation}
with $\epsilon$ a scalar, it is discretized as :
\begin{eqnarray}
 k_{i,j} u_{i,j}+ \frac{\epsilon_{i-,j}(u_{i,j} -u_{i-1,j})+\epsilon_{i+,j}(u_{i,j}-u_{i+1,j})}{h_x^2} + \hspace{2cm}\nonumber \\
\frac{\epsilon_{i,j-}(u_{i,j} -u_{i,j-1})+\epsilon_{i,j+}(u_{i,j}-u_{i,j+1})}{h_y^2}  = f_{i,j}
\end{eqnarray}

If $(i,j)$ is 'red', then $(i-1,j),(i+1,j),(i,j-1),(i,j+1)$ is 'black'.
So that, the linear equation can be written as:
\begin{equation}
\left(
\begin{array}{cc}
D_r & C\\
C^T & D_b
\end{array}
\right)\left(
\begin{array}{c}
u_r\\
u_b
\end{array}
\right)=\left(
\begin{array}{c}
f_r\\
f_b
\end{array}
\right)
\end{equation}
where $D_r$ and $D_b$ are diagonal matrix, $C$ is sparse square matrix.

RBSGS has three steps:
\begin{itemize}
\item[(1)], $D_r u_r^{(n+1/2)} = f_r - C u^{(n)}_b$
\item[(2)], $D_b u_b^{(n+1)} = f_b - C^T u^{(n+1/2)}_r$
\item[(3)], $D_r u_r^{(n+1)} = f_r - C u^{(n+1)}_b$
\end{itemize}
In matrix form:
\begin{equation}
\left(
\begin{array}{cc}
D_r & C\\
C^T & D_b+C^TD_r^{-1}C
\end{array}
\right)\left(
\begin{array}{c}
u_r^{(n+1)}\\
u_b^{(n+1)}
\end{array}
\right)=\left(
\begin{array}{c}
f_r\\
f_b
\end{array}
\right)+\left(\begin{array}{cc}
0 & 0\\
0 & C^TD_r^{-1}C
\end{array}
\right)\left(
\begin{array}{c}
u_r^{(n)}\\
u_b^{(n)}
\end{array}
\right)
\end{equation}

\subsection{Projection}
To ensure the preconditioner is symmetric, we could first require the restriction operator $r$ from fine grid to coarse grid is the transpose(adjoint) of the prolongation operator $p$ from coarse grid to fine grid: $r=b p^T$ where $b$ is a constant to ensure $r*p=1$.

\subsection{MG circle}
We could use V-circle or W-circle. The resulting MGCG satisfy the condition we need.

Full-circle????NOT SURE


\section{Advantage}
The CG iteration steps is independent with grid size. 





\end{document}
